\documentclass[10pt, oneside]{article}
\usepackage{amsmath, amsthm, amssymb, calrsfs, wasysym, verbatim, bbm, color, graphics, geometry, cite}
\geometry{tmargin=.75in, bmargin=.75in, lmargin=.75in, rmargin = .75in}







\newcommand{\R}{\mathbb{R}}
\newcommand{\C}{\mathbb{C}}
\newcommand{\Z}{\mathbb{Z}}
\newcommand{\N}{\mathbb{N}}
\newcommand{\Q}{\mathbb{Q}}
\newcommand{\Cdot}{\boldsymbol{\cdot}}
\newcommand{\F}{\mathfrak{F}}










\newtheorem{thm}{Theorem}
\newtheorem{defn}{Definition}
\newtheorem{conv}{Convention}
\newtheorem{rem}{Remark}
\newtheorem{lem}{Lemma}
\newtheorem{cor}{Corollary}
\newtheorem{example}{Example}
\newtheorem{exe}{Exercise}
\newtheorem{conjecture}{Conjecture}
\newtheorem{remark}{Remark}
\title{A Practical Application of the Banach-Tarski Paradox to Fluid Dynamics?}
\author{[Drew Remmenga]}
\begin{document}

\maketitle
\begin{abstract}
   We postulate on a connection between blow up times of Navier-Stokes equations and the Banach-Tarski Paradox.
\end{abstract}
\section*{Observations}
   If we plug our sink, turn on the faucet to a critical value, and watch the fluid evolve away from a single body given by a manifold in the reservoir $M_r$. It evolves through the tap and dribbles into multiple manifolds $M_1 \cdots M_i$ and into the sink where the submanifolds merge $M_s$.
   We turn off the sink and the fluid rests in the reservoir and in the sink. $M_r \to M_r + M_1 + \cdots + M_n + M_s \to M_r + M_s$. As topology this looks like nonsense. We will try to identify a way to talk about this phenomenon.
\section*{Existing Methods}
\subsection*{Navier-Stokes}
   Navier-Stokes equations blow up in finite time \cite{tao2015finitetimeblowupaveraged} and have lots of restrictions which I will not repeat but can be found here \cite{tao2015finitetimeblowupaveraged}. This blow up is not a bug but a feature, more on that later.
   I'm only going to make the problem worse with more restrictions. Let's denote the set of solutions to these equations on a manifold $M$ as $F(M)$.
\subsection*{Volume Preserving Diffeomorphism}
   The group of volume preserving diffeomorphism of smoothness $k$ between homeotopic manifolds is denoted $\text{Diff}^k_\omega(M)$. As a group it is perfect and as a Lie Algebra it is infinite dimensional \cite{Banyaga1997}. We tend to favor the infinitely smooth diffeomorphisms $k \to \infty$.
   Additionally, instead of volume form $\omega$ preserving diffeomorphisms in this case we want the mass preserving ones denoted with an $m$. $m$ is defined the same as in the Navier-Stokes Equations. This group denoted by $\text{Diff}^k_m(M)$ is probably isomorphic to $\text{Diff}^k_\omega(M)$. \footnote{I very badly want to put a citation here.}
   We use mass because mass is conserved even in compressible fluids.
\section*{Banach Tarski}
   The Banach-Tarski paradox states that as a consequence of the axiom of choice one can disassemble a unit ball into discrete subsets and reassemble these subsets into two new unit balls \cite{Tao2011}. We know that simples - molecules - in a fluid can be interchanged. Fundamentally two water molecules are isomorphic, the water in our reservoir $M_r$ can probably be considered well mixed. And we can select a molecule of water and move it into the new subsets in general.
   We denote the $split$ of a manifold $M$ using the Banach-Tarski result as $M|^\wedge_+$ yielding two new manifolds $M_1, M_2$. We can impose an additional restriction on the manifolds such that mass\footnote{Volume form may work just as .} is conserved. $(M_1,m_1)|^{wedge}_+ = (M_2,m_2) \oplus (M_3,m_3)$. Where $m_1 = m_2 + m_3$. In further notation we will drop the little $m$'s and take this result as part of the operation $|^\wedge_+$.
   As far as I am aware there is no set theory result to reverse this operation, but we will denote it $|^\wedge_-$. Which takes in two manifolds $M_1$ and $M_2$ and yields a third manifold $M_3$ where mass is once again conserved.
   There may be something clever one can do with Ricci flow here to dissolve one of the two smaller manifolds as time evolves.\cite{gianniotis2015ricciflowmanifoldsboundary} As a whole we refer to all these manifolds under these operations $\mathfrak{M} = (M, |^\wedge)$
\section*{Set of Solutions}
   So if we want to bound $F(M)$ we can take the intersections of $\F(M) = F(M) \cap \text{Diff}_m^\infty(M)$ \footnote{It may also work to intersect solutions to Navier-Stokes with $C^k$ functions.} Furthermore we want our manifolds $M$ to $split$ and $merge$. We want to talk about the total space $X$ with its own topology $\mathcal{O}$ with the submanifolds embedded within that topology $\mathfrak{M}$.
   Then the totality of fluid solutions can be written $(X,\mathcal{O},\F(\mathfrak{M}))$. As solutions $F(M)$ become unbounded at time $t$ it may be possible to drop the intersection with $\text{Diff}_m^\infty (M)$ and link the time $t$ to a $split$.
\section*{Further Research}
   Couple of open questions:
   \begin{align}
       \text{Diff}^k_m(M)\cong \text{Diff}_\omega^k (M)
   \end{align}
   In set theory:
   \begin{align}
       \exists |^{\wedge}_-
   \end{align}
   A critical point in time $t$:
   \begin{align}
       \text{Ricci Flow}(t) \propto |^{wedge}_-(t)
   \end{align}
   \begin{align}
       \text{sup}(|F(m)|)(t) \propto |^{wedge}_+ (t)
   \end{align}
   Finally: is the total topology $(X,\mathcal{O})$ well defined?
\bibliographystyle{plain}  % or another style like alpha, unsrt, etc.
\bibliography{references.bib}  % the name of the .bib file
\end{document}